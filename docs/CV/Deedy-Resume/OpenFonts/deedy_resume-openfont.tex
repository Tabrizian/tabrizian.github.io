%%%%%%%%%%%%%%%%%%%%%%%%%%%%%%%%%%%%%%%
% Deedy - One Page Two Column Resume
% LaTeX Template
% Version 1.2 (16/9/2014)
%
% Original author:
% Debarghya Das (http://debarghyadas.com)
%
% Original repository:
% https://github.com/deedydas/Deedy-Resume
%
% IMPORTANT: THIS TEMPLATE NEEDS TO BE COMPILED WITH XeLaTeX
%
% This template uses several fonts not included with Windows/Linux by
% default. If you get compilation errors saying a font is missing, find the line
% on which the font is used and either change it to a font included with your
% operating system or comment the line out to use the default font.
%
%%%%%%%%%%%%%%%%%%%%%%%%%%%%%%%%%%%%%%
%
% TODO:
% 1. Integrate biber/bibtex for article citation under publications.
% 2. Figure out a smoother way for the document to flow onto the next page.
% 3. Add styling information for a "Projects/Hacks" section.
% 4. Add location/address information
% 5. Merge OpenFont and MacFonts as a single sty with options.
%%%%%%%%%%%%%%%%%%%%%%%%%%%%%%%%%%%%%%
%
%
% CHANGELOG:
% v1.1:
% 1. Fixed several compilation bugs with \renewcommand
% 2. Got Open-source fonts (Windows/Linux support)
% 3. Added Last Updated
% 4. Move Title styling into .sty
% 5. Commented .sty file.
%
%%%%%%%%%%%%%%%%%%%%%%%%%%%%%%%%%%%%%%%
%
% Known Issues:
% 1. Overflows onto second page if any column's contents are more than the
% vertical limit
% 2. Hacky space on the first bullet point on the second column.
%
%%%%%%%%%%%%%%%%%%%%%%%%%%%%%%%%%%%%%%


\documentclass[]{deedy-resume-openfont}
\usepackage{fancyhdr}

\pagestyle{fancy}
\fancyhf{}

\begin{document}

%%%%%%%%%%%%%%%%%%%%%%%%%%%%%%%%%%%%%%
%
%     LAST UPDATED DATE
%
%%%%%%%%%%%%%%%%%%%%%%%%%%%%%%%%%%%%%%
\lastupdated

%%%%%%%%%%%%%%%%%%%%%%%%%%%%%%%%%%%%%%
%
%     TITLE NAME
%
%%%%%%%%%%%%%%%%%%%%%%%%%%%%%%%%%%%%%%
\namesection{Iman}{Tabrizian}{ \
\href{mailto:iman.tabrizian@mail.utoronto.ca}{iman.tabrizian@mail.utoronto.ca} | +1 (647) 551-9065 | \href{mailto:iman.tabrizian@gmail.com}{iman.tabrizian@gmail.com}
}

%%%%%%%%%%%%%%%%%%%%%%%%%%%%%%%%%%%%%%
%
%     COLUMN ONE
%
%%%%%%%%%%%%%%%%%%%%%%%%%%%%%%%%%%%%%%

\begin{minipage}[t]{0.33\textwidth}

%%%%%%%%%%%%%%%%%%%%%%%%%%%%%%%%%%%%%%
%     EDUCATION
%%%%%%%%%%%%%%%%%%%%%%%%%%%%%%%%%%%%%%

\section{Education}

\subsection{University of Toronto}
\descript{MASc in Electrical and Computer Engineering}
    \location{2019 - 2020 (Expected) | Tehran, Iran}
\subsection{Amirkabir University}
\descript{BS in Computer Engineering}
    \location{2014 - 2018 | Tehran, Iran}
    \location{ Cum. GPA: 17.97 / 20}

\sectionsep

%%%%%%%%%%%%%%%%%%%%%%%%%%%%%%%%%%%%%%
%     RESEARCH INTRESTS
%%%%%%%%%%%%%%%%%%%%%%%%%%%%%%%%%%%%%%
\section{Research Interests}
Internet of Things \\
Cloud Orchestration \\
Machine Learning

\sectionsep
%%%%%%%%%%%%%%%%%%%%%%%%%%%%%%%%%%%%%%
%     LINKS
%%%%%%%%%%%%%%%%%%%%%%%%%%%%%%%%%%%%%%

\section{Links}
Github:// \href{https://github.com/Tabrizian}{\bf Tabrizian} \\
LinkedIn://  \href{https://www.linkedin.com/in/tabrizian}{\bf tabrizian} \\
StackOverflow://  \href{https://stackoverflow.com/users/4453461/iman}{\bf iman}

%%%%%%%%%%%%%%%%%%%%%%%%%%%%%%%%%%%%%%%
%%     COURSEWORK
%%%%%%%%%%%%%%%%%%%%%%%%%%%%%%%%%%%%%%%
%
%\section{Coursework}
%\subsection{Undergraduate}
%Computer Networking (18.3 / 20)\\
%    Operating Systems (17.9 / 20)\\
%Data Structures \& Algorithms (20 / 20)\\
%Algorithm Design (20 / 20)\\
%Theory of Machines \& Languages (18.5 / 20)\\
%Engineering Statistics (19.7 / 20)\\
%{\footnotesize \textit{\textbf{(Teaching Asst 3x) }}} \\

%%%%%%%%%%%%%%%%%%%%%%%%%%%%%%%%%%%%%%
%     SKILLS
%%%%%%%%%%%%%%%%%%%%%%%%%%%%%%%%%%%%%%

\section{Skills}
\subsection{Programming}
\location{Over 5000 lines:}
Java \textbullet{}  Python \textbullet{} Javascript \\
Android \textbullet{}  Node.js  \\
 C \textbullet{} C++ \textbullet{} VHDL \textbullet{}  PHP \\


\location{Over 1000 lines:}
CSS \textbullet{} Assembly \\
\location{Familiar:}
MySQL \textbullet{} Verilog \textbullet{} MongoDB \\

\sectionsep

\subsection{Frameworks}
Kubernetes \textbullet{} Docker \textbullet{} Swarm \\
Kaa IoT Platform \\ Prometheus \textbullet{} \\
OpenStack \textbullet{} Open vSwtich \textbullet{} \\
Mininet \textbullet{} Hapi.js \textbullet{} Vue.js
\sectionsep


\section{Awards}
    \vspace{\topsep} % Hacky fix for awkward extra vertical space
\begin{tightemize}
\item Selected to Participate in \\
	Facebook PyTorch Scholarship
\item top 10\% in terms of cumalative \\ G.P.A (2018)
\item Offered direct admission to \\ continue graduate studies (2018)
\item Eliglbe to choose second major (2018)
\item Ranked $1^{st}$ among all \\ entrance university students \\
	in ${14^{th}}$ Amirkabir ACM-ICPC (2014)
\item Top 0.7\% in the national \\university entrance exam (2014)
\end{tightemize}

%%%%%%%%%%%%%%%%%%%%%%%%%%%%%%%%%%%%%%
%
%     COLUMN TWO
%
%%%%%%%%%%%%%%%%%%%%%%%%%%%%%%%%%%%%%%

\end{minipage}
\hfill
\begin{minipage}[t]{0.66\textwidth}

%%%%%%%%%%%%%%%%%%%%%%%%%%%%%%%%%%%%%%
%     RESEARCH & Devlopment Experience
%%%%%%%%%%%%%%%%%%%%%%%%%%%%%%%%%%%%%%

\section{Research \& Development Experience}


\runsubsection{AUT IoT Laborartory}
    \descript{| Researcher \& Developer}
    \location{Jan 2016 – Nov 2016 | Tehran, Iran}
    \vspace{\topsep} % Hacky fix for awkward extra vertical space
    \begin{tightemize}
    \item
        Experimenting with Kaa to provide a Building Automation solution.
        This project used nRFs for providing connectivity for sensors and
        used Raspbery Pi as the gateway which connected to the internet.

    \item
        Development of an IoT platform with the ability to support logging, event
        triggering, scenario creation. It was based on the MQTT protocol
        for the reliable and bidirectional communication between IoT devices.
        Under supervision of \href{http://ceit.aut.ac.ir/~bakhshis}{\textbf{Prof. Bahador Bakhshi}}
    \end{tightemize}
\sectionsep
\runsubsection{Andishe Fartak Amirkabir}
    \descript{| Developer Operations}
    \location{September 2018 – December 2018 | Tehran, Iran}
    \vspace{\topsep} % Hacky fix for awkward extra vertical space
    \begin{tightemize}
    \item Setup a CI / CD pipeline for automatic deployment to Kubernetes using drone.io 
    \item Migrated all of the web applications to Kubernetes
    \item Setup a 3 node Kubernetes Cluster
    \item Setup of Rook as the storage orchestration
    \item Modification of CoreDNS configuration to deal with enterprise proxy
    \end{tightemize}

\runsubsection{Farzan Fan Andish Farda}
    \descript{| Researcher \& Developer}
    \location{Jun 2016 – December 2018 | Tehran, Iran}
    \vspace{\topsep} % Hacky fix for awkward extra vertical space
    \begin{tightemize}
    \item Development of a product for remote management of 250 welding robots
        for IKCO - leading car company in Iran. The solution that we proposed
        simplified the provisioning process extremely and helped them increase
        the number of welding robots in each production line.

    \item R\&D for development of an IoT based assembly platform which eased
        the production of hard to assemble products. This platform used MQTT
        as the messaging broker and Node.js as the backend providing restful
        web services for things connecting to them.
    \end{tightemize}

\sectionsep

%%%%%%%%%%%%%%%%%%%%%%%%%%%%%%%%%%%%%%
%    PROJECTS
%%%%%%%%%%%%%%%%%%%%%%%%%%%%%%%%%%%%%%
\section{Projects}
\runsubsection{Genorch}
    \descript{| Python}
    Development of a generic orchestration platform with the ability to
    auto scale the infrastructure of an application based on the user defined
    criteria. It is written extremely modular and currently supports OpenStack
    It is an open source project you can view the source code: \href{https://github.com/genorch/orchestration}{github.com/genorch/orchestration}
    It was tested on SAVI (testbed of University of Toronto).
\sectionsep

    \runsubsection{An auto-scaler for Docker Swarm}
    \descript{| Python}
    \location{Final year dissertion}
    Developed an auto-scaler for Docker Swarm with support for any criterion. This project used Prometheus as the monitoring and time-series data collection. Acted upon the alertmanager notifications and scaled the Swarm Cluster upward/downward. You can view the project in \href{https://github.com/Tabrizian/swascale}{here}.

\sectionsep

\runsubsection{Bamboo}
    \descript{| Node.js}
    Bamboo is an IoT platform whose architecture is based on microservices.
    I was the architectural designer of bamboo and how components should be
    divided. It is using MQTT message broker for connectivity.
    You can findout more about it \href{https://github.com/bambil/bamboo}{here}.
\sectionsep


%%%%%%%%%%%%%%%%%%%%%%%%%%%%%%%%%%%%%%
%     PUBLICATIONS
%%%%%%%%%%%%%%%%%%%%%%%%%%%%%%%%%%%%%%

% \section{Publications}
% \renewcommand\refname{\vskip -1.5cm} % Couldn't get this working from the .cls file
% \bibliographystyle{abbrv}
% \bibliography{publications}
% \nocite{*}

\end{minipage}
\begin{minipage}[t]{0.33\textwidth}

%%%%%%%%%%%%%%%%%%%%%%%%%%%%%%%%%%%%%%
%
%     COLUMN TWO
%
%%%%%%%%%%%%%%%%%%%%%%%%%%%%%%%%%%%%%%
%%%%%%%%%%%%%%%%%%%%%%%%%%%%%%%%%%%%%%
%     Languages
%%%%%%%%%%%%%%%%%%%%%%%%%%%%%%%%%%%%%%
\section{Languages}
English: professional working \\proficiency (TOEFL iBT 107) \\
Persian: Native proficiency

\section{Teaching}
    \vspace{\topsep} % Hacky fix for awkward extra vertical space
\begin{tightemize}
\item Teaching Assistant 3x
\item Instructor of Programming with C++ \\at Rouzbeh High School for 3 years.
\end{tightemize}


\end{minipage}
\hfill
\begin{minipage}[t]{0.66\textwidth}

\runsubsection{Iran Metro}
    \descript{| Android}
    A transportation application for Iran subway system. In currently
    has more than 20,000 active users. It was selected as the BESTS APP of THE
    WEEK by cafebazaar (Iranian Android Market) when it was launched.
\sectionsep

\runsubsection{Merger}
    \descript{| Node.js}
    This project was developed based on the request from the ICT of Amirkabir
    University of Technology. \textbf{Merger} fetches data from the different
    academic sources namely, elsevier, IEEE and crossref, merges this data
    into a unified format and converts them to appropriate database models.
\sectionsep

%%%%%%%%%%%%%%%%%%%%%%%%%%%%%%%%%%%%%%
%     TEACHING EXPERIENCE
%%%%%%%%%%%%%%%%%%%%%%%%%%%%%%%%%%%%%%

%%%%%%%%%%%%%%%%%%%%%%%%%%%%%%%%%%%%%%
%     AWARDS
%%%%%%%%%%%%%%%%%%%%%%%%%%%%%%%%%%%%%%

% \section{Awards}
% \begin{tabular}{rll}
% 2017	     & top 10 \% &  in terms of cumalative G.P.A among all the students \\
% 2017	     & &  Offered direct admission to continue graduate studies without exam\\
% 2015	     & &  Eligble to choose second major due to outstanding performance\\
% 2014	     & 1\textsuperscript{st}  & among all entrance students in $14^{th}$ Amirkabir ACM-ICPC \\
% 2014	     & top 0.007 \%  & in the national university entrance exam\\
% \end{tabular}
% \sectionsep
%%%%%%%%%%%%%%%%%%%%%%%%%%%%%%%%%%%%%%
%     Extra Curricular Activities
%%%%%%%%%%%%%%%%%%%%%%%%%%%%%%%%%%%%%%

\section{Extra Curricular Activities}
\runsubsection{OpenIoT Summit North America}
\descript{| Speaker}
\location{2018 | Portland, Oregon}
I was invited to speak at OpenIoT Summit North America about On the Air Firmware
Update Using MQTT.

\sectionsep
\runsubsection{Undergrad Talks}
\descript{| Speaker}
\location{2018 | Amirkabir University of Technology}
    I had a 40 minutes talk with the title \textbf{What is Cloud Orchestration?}.
    This presentation included the current approach to cloud orchestration and
    presentation of some infrastructure as code templates.

\sectionsep
\runsubsection{8th Linux Festival}
\descript{| Virtualization Workshop}
\location{2017 | Amirkabir University of Technology}
I was the instructor about the virtualization technologies in general
and how to use Docker specifically for containerization.
\sectionsep

\runsubsection{Node.js Summer Course}
\descript{| Instructor}
\location{Summer 2017 | Computer Departemnt Scientific Committee}
Teaching Node.js basics from the ground to web application and bot development.
\sectionsep

\runsubsection{7th Linux Festival}
\descript{| Linux Basics Presenter}
\location{2016 | Amirkabir University of Technology}
I had a 20 minutes talk about code editing in Linux and about
best practices in code editing.

\sectionsep





\sectionsep
%%%%%%%%%%%%%%%%%%%%%%%%%%%%%%%%%%%%%%
%     REFERENCES
%%%%%%%%%%%%%%%%%%%%%%%%%%%%%%%%%%%%%%

\section{References}
\runsubsection{Bahador Bakhshi}
    \descript{|  Assistance Professor}
    Amirkabir University of Technology, Department of Computer Engineering and Information Technology \\
    Email: \href{mailto:bbakhshi@aut.ac.ir}{bbakhshi@aut.ac.ir}
\sectionsep

\runsubsection{Mehdi Dehghan}
    \descript{| Professor}
    Amirkabir University of Technology, Department of Computer Engineering and Information Technology \\
    Email: \href{mailto:dehghan@aut.ac.ir}{dehghan@aut.ac.ir}
\sectionsep

\runsubsection{Masoud Sabaei}
    \descript{| Associate Professor}
    Amirkabir University of Technology, Department of Computer Engineering and Information Technology \\
    Email: \href{mailto:sabaei@aut.ac.ir}{sabaei@aut.ac.ir}
\sectionsep

% \section{Publications}
% \renewcommand\refname{\vskip -1.5cm} % Couldn't get this working from the .cls file
% \bibliographystyle{abbrv}
% \bibliography{publications}
% \nocite{*}

\end{minipage}
\end{document}  \documentclass[]{article}
