%%%%%%%%%%%%%%%%%%%%%%%%%%%%%%%%%%%%%%%
% Deedy - One Page Two Column Resume
% LaTeX Template
% Version 1.2 (16/9/2014)
%
% Original author:
% Debarghya Das (http://debarghyadas.com)
%
% Original repository:
% https://github.com/deedydas/Deedy-Resume
%
% IMPORTANT: THIS TEMPLATE NEEDS TO BE COMPILED WITH XeLaTeX
%
% This template uses several fonts not included with Windows/Linux by
% default. If you get compilation errors saying a font is missing, find the line
% on which the font is used and either change it to a font included with your
% operating system or comment the line out to use the default font.
%
%%%%%%%%%%%%%%%%%%%%%%%%%%%%%%%%%%%%%%
%
% TODO:
% 1. Integrate biber/bibtex for article citation under publications.
% 2. Figure out a smoother way for the document to flow onto the next page.
% 3. Add styling information for a "Projects/Hacks" section.
% 4. Add location/address information
% 5. Merge OpenFont and MacFonts as a single sty with options.
%
%%%%%%%%%%%%%%%%%%%%%%%%%%%%%%%%%%%%%%
%
% CHANGELOG:
% v1.1:
% 1. Fixed several compilation bugs with \renewcommand
% 2. Got Open-source fonts (Windows/Linux support)
% 3. Added Last Updated
% 4. Move Title styling into .sty
% 5. Commented .sty file.
%
%%%%%%%%%%%%%%%%%%%%%%%%%%%%%%%%%%%%%%%
%
% Known Issues:
% 1. Overflows onto second page if any column's contents are more than the
% vertical limit
% 2. Hacky space on the first bullet point on the second column.
%
%%%%%%%%%%%%%%%%%%%%%%%%%%%%%%%%%%%%%%


\documentclass[]{deedy-resume-openfont}
\usepackage{fancyhdr}

\pagestyle{fancy}
\fancyhf{}

\begin{document}

%%%%%%%%%%%%%%%%%%%%%%%%%%%%%%%%%%%%%%
%
%     LAST UPDATED DATE
%
%%%%%%%%%%%%%%%%%%%%%%%%%%%%%%%%%%%%%%
\lastupdated

%%%%%%%%%%%%%%%%%%%%%%%%%%%%%%%%%%%%%%
%
%     TITLE NAME
%
%%%%%%%%%%%%%%%%%%%%%%%%%%%%%%%%%%%%%%
\namesection{Iman}{Tabrizian}{ \urlstyle{same}\href{http://imantabrizian.me}{imantabrizian.me} | \href{http://ceit.aut.ac.ir/~tabrizian}{ceit.aut.ac.ir/\~tabrizian}\\
\href{mailto:tabrizian@aut.ac.ir}{tabrizian@aut.ac.ir} | +989377371367 | \href{mailto:iman.tabrizian@gmail.com}{iman.tabrizian@gmail.com}
}

%%%%%%%%%%%%%%%%%%%%%%%%%%%%%%%%%%%%%%
%
%     COLUMN ONE
%
%%%%%%%%%%%%%%%%%%%%%%%%%%%%%%%%%%%%%%

\begin{minipage}[t]{0.33\textwidth}

%%%%%%%%%%%%%%%%%%%%%%%%%%%%%%%%%%%%%%
%     EDUCATION
%%%%%%%%%%%%%%%%%%%%%%%%%%%%%%%%%%%%%%

\section{Education}

\subsection{Amirkabir University}
\descript{BS in Computer Engineering}
    \location{2014 - 2018 (Expected) | Tehran, Iran}
    \location{ Cum. GPA: 18.13 / 20 \textbf{or} 3.78 / 4.0}

\sectionsep

%%%%%%%%%%%%%%%%%%%%%%%%%%%%%%%%%%%%%%
%     RESEARCH INTRESTS
%%%%%%%%%%%%%%%%%%%%%%%%%%%%%%%%%%%%%%
\section{Research Interests}
Internet of Things \\
Cloud Orchestration \\
Software Defined Networking

\sectionsep
%%%%%%%%%%%%%%%%%%%%%%%%%%%%%%%%%%%%%%
%     LINKS
%%%%%%%%%%%%%%%%%%%%%%%%%%%%%%%%%%%%%%

\section{Links}
Github:// \href{https://github.com/Tabrizian}{\bf Tabrizian} \\
LinkedIn://  \href{https://www.linkedin.com/in/tabrizian}{\bf tabrizian} \\
StackOverflow://  \href{https://stackoverflow.com/users/4453461/iman}{\bf iman}

%%%%%%%%%%%%%%%%%%%%%%%%%%%%%%%%%%%%%%
%     COURSEWORK
%%%%%%%%%%%%%%%%%%%%%%%%%%%%%%%%%%%%%%

\section{Coursework}
\subsection{Undergraduate}
Computer Networking (18.3 / 20)\\
Operating Systems \\
Data Structures \& Algorithms (20 / 20)\\
Algorithm Design (20 / 20)\\
Theory of Machines \& Languages (18.5 / 20)\\
Engineering Statistics (19.7 / 20)\\
{\footnotesize \textit{\textbf{(Teaching Asst 3x) }}} \\

%%%%%%%%%%%%%%%%%%%%%%%%%%%%%%%%%%%%%%
%     SKILLS
%%%%%%%%%%%%%%%%%%%%%%%%%%%%%%%%%%%%%%

\section{Skills}
\subsection{Programming}
\location{Over 5000 lines:}
Java \textbullet{}  Python \textbullet{} Javascript \\
Android \textbullet{}  Node.js  \\
 C \textbullet{} C++ \\
VHDL \textbullet{}  PHP \\


\location{Over 1000 lines:}
CSS \textbullet{} Assembly \\
\location{Familiar:}
MySQL \textbullet{} Verilog \textbullet{} MongoDB \\

\sectionsep

\subsection{Frameworks}
ONOS \textbullet{} Docker \textbullet{} Swarm \\
Kaa IoT Platform \textbullet{} Lelylan IoT Platform \textbullet{} \\
OpenStack \textbullet{} Open vSwtich \textbullet{} \\
Mininet \textbullet{} Hapi.js \textbullet{} Vue.js
\sectionsep

%%%%%%%%%%%%%%%%%%%%%%%%%%%%%%%%%%%%%%
%     Languages
%%%%%%%%%%%%%%%%%%%%%%%%%%%%%%%%%%%%%%
\section{Languages}
English: professional working proficiency (TOEFL iBT 107) \\
Persian: Native proficiency

%%%%%%%%%%%%%%%%%%%%%%%%%%%%%%%%%%%%%%
%
%     COLUMN TWO
%
%%%%%%%%%%%%%%%%%%%%%%%%%%%%%%%%%%%%%%

\end{minipage}
\hfill
\begin{minipage}[t]{0.66\textwidth}

%%%%%%%%%%%%%%%%%%%%%%%%%%%%%%%%%%%%%%
%     RESEARCH & Devlopment Experience
%%%%%%%%%%%%%%%%%%%%%%%%%%%%%%%%%%%%%%

\section{Research \& Development Experience}
\runsubsection{Farzan Fan Andish Farda}
    \descript{| IOT Researcher \& Developer}
    \location{Jun 2016 – Current | Tehran, Iran}
    \vspace{\topsep} % Hacky fix for awkward extra vertical space
    \begin{tightemize}
    \item Development of a product for remote management of 250 welding robots
        for IKCO - leading car company in Iran. The solution that we proposed
        simplified the provisioning process extremely and helped them increase
        the number of welding robots in each production line.

    \item R\&D for development of an IoT based assembly platform which eased
        the production of hard to assemble products. This platform used MQTT
        as the messaging broker and Node.js as the backend providing restful
        web services for things connecting to them.
    \end{tightemize}

\sectionsep

\runsubsection{University of Toronto}
    \descript{| Researcher}
    \location{Jun 2016 – Current | Tehran, Iran}
    Development of a generic orchestration platform with the ability to
    auto scale the infrastructure of an application based on the user defined
    criteria. It is written extremely modular and currently supports OpenStack
    It is an open source project you can view the source code: \href{https://github.com/genorch/orchestration}{github.com/genorch/orchestration}
\sectionsep

\runsubsection{ICT of Amirkabir University of Technology}
    \descript{| Developer}
    \location{Jun 2016 – Nov 2016 | Tehran, Iran}
    Development of a system to automatically fetch meta data about
    scientifitic papers from different sources and merge the data into a
    specified format.
\sectionsep

\runsubsection{AUT IoT Laborartory}
    \descript{| Researcher \& Developer}
    \location{Jan 2016 – Nov 2016 | Tehran, Iran}
    Development of an IoT platform with the ability to support logging, event
    triggering, scenario creation. It was based on the MQTT protocol
    for the reliable and bidirectional communication between IoT devices.
    Under supervision of \href{http://ceit.aut.ac.ir/~bakhshis}{\textbf{Prof. Bahador Bakhshi}}
\sectionsep

\section{Extra Curricular Activities}
\runsubsection{8th Linux Festival}
\descript{| Virtualization Workshop}
\location{2017 | Amirkabir University of Technology}
I was the instructor about the virtualization technologies in general
and how to use Docker specifically for containerization.
\sectionsep

\runsubsection{Node.js Summer Course}
\descript{| Instructor}
\location{Summer 2017 | Computer Departemnt Scientifitic Committee}
Teaching Node.js basics from the ground to web application and bot development.
\sectionsep

\runsubsection{7th Linux Festival}
\descript{| Linux Basics Presenter}
\location{2016 | Amirkabir University of Technology}
I had a 20 minutes talk about code editing in Linux and about
best practices in code editing.

\sectionsep
%%%%%%%%%%%%%%%%%%%%%%%%%%%%%%%%%%%%%%
%     AWARDS
%%%%%%%%%%%%%%%%%%%%%%%%%%%%%%%%%%%%%%

\section{Awards}
\begin{tabular}{rll}
2017	     & top 10 \% &  in terms of cumalative G.P.A among all the students \\
2017	     & &  Offered direct admission to continue graduate studies without exam\\
2015	     & &  Eligble to choose second major due to outstanding performance\\
2014	     & 1\textsuperscript{st}  & among all entrance students in $14^{th}$ Amirkabir ACM-ICPC \\
2014	     & top 0.007 \%  & in the national university entrance exam\\
\end{tabular}
\sectionsep

%%%%%%%%%%%%%%%%%%%%%%%%%%%%%%%%%%%%%%
%     PUBLICATIONS
%%%%%%%%%%%%%%%%%%%%%%%%%%%%%%%%%%%%%%

% \section{Publications}
% \renewcommand\refname{\vskip -1.5cm} % Couldn't get this working from the .cls file
% \bibliographystyle{abbrv}
% \bibliography{publications}
% \nocite{*}

\end{minipage}
\end{document}  \documentclass[]{article}
