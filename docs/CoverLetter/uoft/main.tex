%% start of file `template.tex'.
%% Copyright 2006-2013 Xavier Danaux (xdanaux@gmail.com).
%
% This work may be distributed and/or modified under the
% conditions of the LaTeX Project Public License version 1.3c,
% available at http://www.latex-project.org/lppl/.


\documentclass[11pt,a4paper,sans]{moderncv}        % possible options include font size ('10pt', '11pt' and '12pt'), paper size ('a4paper', 'letterpaper', 'a5paper', 'legalpaper', 'executivepaper' and 'landscape') and font family ('sans' and 'roman')

% moderncv themes
\moderncvstyle{classic}                            % style options are 'casual' (default), 'classic', 'oldstyle' and 'banking'
\moderncvcolor{green}                              % color options 'blue' (default), 'orange', 'green', 'red', 'purple', 'grey' and 'black'
%\renewcommand{\familydefault}{\sfdefault}         % to set the default font; use '\sfdefault' for the default sans serif font, '\rmdefault' for the default roman one, or any tex font name
%\nopagenumbers{}                                  % uncomment to suppress automatic page numbering for CVs longer than one page

% character encoding
\usepackage[utf8]{inputenc}                       % if you are not using xelatex ou lualatex, replace by the encoding you are using
%\usepackage{CJKutf8}                              % if you need to use CJK to typeset your resume in Chinese, Japanese or Korean

% adjust the page margins
\usepackage[scale=0.75]{geometry}
%\setlength{\hintscolumnwidth}{3cm}                % if you want to change the width of the column with the dates
%\setlength{\makecvtitlenamewidth}{10cm}           % for the 'classic' style, if you want to force the width allocated to your name and avoid line breaks. be careful though, the length is normally calculated to avoid any overlap with your personal info; use this at your own typographical risks...

% personal data
\name{Iman}{Tabrizian}
\title{Letter of Motivation}                               % optional, remove / comment the line if not wanted
%\address{street and number}{postcode city}{country}% optional, remove / comment the line if not wanted; the "postcode city" and and "country" arguments can be omitted or provided empty
\phone[mobile]{+989377371367}                   % optional, remove / comment the line if not wanted
%\phone[fixed]{+2~(345)~678~901}                    % optional, remove / comment the line if not wanted
%\phone[fax]{+3~(456)~789~012}                      % optional, remove / comment the line if not wanted
\email{tabrizian@aut.ac.ir}                               % optional, remove / comment the line if not wanted
\homepage{imantabrizian.me}                         % optional, remove / comment the line if not wanted
%\extrainfo{additional information}                 % optional, remove / comment the line if not wanted
%\photo[64pt][0.4pt]{picture}                       % optional, remove / comment the line if not wanted; '64pt' is the height the picture must be resized to, 0.4pt is the thickness of the frame around it (put it to 0pt for no frame) and 'picture' is the name of the picture file
%\quote{Some quote}                                 % optional, remove / comment the line if not wanted

% to show numerical labels in the bibliography (default is to show no labels); only useful if you make citations in your resume
%\makeatletter
%\renewcommand*{\bibliographyitemlabel}{\@biblabel{\arabic{enumiv}}}
%\makeatother
%\renewcommand*{\bibliographyitemlabel}{[\arabic{enumiv}]}% CONSIDER REPLACING THE ABOVE BY THIS

% bibliography with mutiple entries
%\usepackage{multibib}
%\newcites{book,misc}{{Books},{Others}}
%----------------------------------------------------------------------------------
%            content
%----------------------------------------------------------------------------------
\begin{document}
%-----       letter       ---------------------------------------------------------
% recipient data
\recipient{University of Toronto}{Electrical \& Computer Engineering Department}
\date{November 4, 2017}
\opening{}
\closing{Yours faithfully,}
\enclosure[Attached]{curriculum vit\ae{}}          % use an optional argument to use a string other than "Enclosure", or redefine \enclname

\makelettertitle

First of all, let me introduce myself. I'm Iman Tabrizian undergraduate student
from Amirkabir University of Technology. My purpose is to expand my knowledge
in the area of computer networking. I've been in the R\&D team of IoT Laborartory
for developing a general purpose IoT platfrom. In here we managed to create
a platform based on MQTT for message broker. This platform supported definition
of types in a hierarchical model.

To talk more about my experience in the field of IoT I've worked with a
company in the innovation center of our university to provide the IKCO,
leading car company in Iran, an IoT based solution for manufacturing cars
and the subsystems of the car with less effort and mistake. It is deployed now
and 50 things are connected simultanously. It uses MQTT as its message broker.
I've also worked with University of Toronto and their testbed (SAVI) to deploy
and test a general orchestration platform. This platform is quit modular
and can support different infrastructure as a service to provide its services
to end users. It supports different scaling factors as its criteria for scaling
infrastructure based on dynamic condittions.

Because of my experience in the creation of platform for IoT I've realized that
the goal to make everything software defined is the purpose and future of
research in this field. I understand that softwarization is the
goal no matter what is the area. It may by the real world which will lead
to development of IoT platforms or network which will lead to SDN and their
related platforms.

One of the most important things about my choice for a university for graduate
studies is wealth. Because wealth is required to provide the foundation and
a computing power to allow researchers to really test their ideas. One the
greatness of University of Toronto is that not only it has the wealth, but also
it has SAVI a really great tool that allows researchers to really test their
ideas on real infrastructure. Moreover it has rich faculty members. At our
university we're actually using the text books written by professors at University
of Toronto as a reference text book for some our courses.

For career, I have not currently decided to whether continue my studies to
Phd or join the workforce. I think this program will help me alot to determine
the path for my future. Currently, I'm really interested in the fields of
Computer Networking and especially IoT and SDN, however I'm not sure to what
extent I'm willing to continue my research but I know that I like to continue my
graduate studies at least for MSc.

I believe my grades (3.78 / 4), my work experience and huge dedication
to open source projects distinguishes me from others. I have been in the
industry and I know their challanges so I can better provide solutions that
both help academics grow and also help the industry grow. Moreover, I have been
in the room. I was teaching assistant of 3 professors and I have worked with
2 professors closely at my university. Finally, I have also already worked
with SAVI and I know the challenges that we're currently facing.

\makeletterclosing

\end{document}

%% end of file `template.tex'.
