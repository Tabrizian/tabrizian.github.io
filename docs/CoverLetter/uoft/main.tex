%% start of file `template.tex'.
%% Copyright 2006-2013 Xavier Danaux (xdanaux@gmail.com).
%
% This work may be distributed and/or modified under the
% conditions of the LaTeX Project Public License version 1.3c,
% available at http://www.latex-project.org/lppl/.


\documentclass[11pt,a4paper,sans]{moderncv}        % possible options include font size ('10pt', '11pt' and '12pt'), paper size ('a4paper', 'letterpaper', 'a5paper', 'legalpaper', 'executivepaper' and 'landscape') and font family ('sans' and 'roman')

% moderncv themes
\moderncvstyle{classic}                            % style options are 'casual' (default), 'classic', 'oldstyle' and 'banking'
\moderncvcolor{green}                              % color options 'blue' (default), 'orange', 'green', 'red', 'purple', 'grey' and 'black'
%\renewcommand{\familydefault}{\sfdefault}         % to set the default font; use '\sfdefault' for the default sans serif font, '\rmdefault' for the default roman one, or any tex font name
%\nopagenumbers{}                                  % uncomment to suppress automatic page numbering for CVs longer than one page

% character encoding
\usepackage[utf8]{inputenc}                       % if you are not using xelatex ou lualatex, replace by the encoding you are using
%\usepackage{CJKutf8}                              % if you need to use CJK to typeset your resume in Chinese, Japanese or Korean

% adjust the page margins
\usepackage[scale=0.75]{geometry}
%\setlength{\hintscolumnwidth}{3cm}                % if you want to change the width of the column with the dates
%\setlength{\makecvtitlenamewidth}{10cm}           % for the 'classic' style, if you want to force the width allocated to your name and avoid line breaks. be careful though, the length is normally calculated to avoid any overlap with your personal info; use this at your own typographical risks...

% personal data
\name{Iman}{Tabrizian}
\title{Letter of Motivation}                               % optional, remove / comment the line if not wanted
%\address{street and number}{postcode city}{country}% optional, remove / comment the line if not wanted; the "postcode city" and and "country" arguments can be omitted or provided empty
\phone[mobile]{+989377371367}                   % optional, remove / comment the line if not wanted
%\phone[fixed]{+2~(345)~678~901}                    % optional, remove / comment the line if not wanted
%\phone[fax]{+3~(456)~789~012}                      % optional, remove / comment the line if not wanted
\email{tabrizian@aut.ac.ir}                               % optional, remove / comment the line if not wanted
\homepage{imantabrizian.me}                         % optional, remove / comment the line if not wanted
%\extrainfo{additional information}                 % optional, remove / comment the line if not wanted
%\photo[64pt][0.4pt]{picture}                       % optional, remove / comment the line if not wanted; '64pt' is the height the picture must be resized to, 0.4pt is the thickness of the frame around it (put it to 0pt for no frame) and 'picture' is the name of the picture file
%\quote{Some quote}                                 % optional, remove / comment the line if not wanted

% to show numerical labels in the bibliography (default is to show no labels); only useful if you make citations in your resume
%\makeatletter
%\renewcommand*{\bibliographyitemlabel}{\@biblabel{\arabic{enumiv}}}
%\makeatother
%\renewcommand*{\bibliographyitemlabel}{[\arabic{enumiv}]}% CONSIDER REPLACING THE ABOVE BY THIS

% bibliography with mutiple entries
%\usepackage{multibib}
%\newcites{book,misc}{{Books},{Others}}
%----------------------------------------------------------------------------------
%            content
%----------------------------------------------------------------------------------
\begin{document}
%-----       letter       ---------------------------------------------------------
% recipient data
\recipient{University of Toronto}{Electrical \& Computer Engineering Department}
\date{October 25, 2017}
\opening{}
\closing{Yours faithfully,}
\enclosure[Attached]{curriculum vit\ae{}}          % use an optional argument to use a string other than "Enclosure", or redefine \enclname

\makelettertitle

First of all, let me introduce myself. I'm Iman Tabrizian undergraduate student
from Amirkabir University of Technology. My purpose is to expand my knowledge
and research in the area of computer networking. When I was a children I used
to install Operating System for all of our relatives and families which shows
an innate eagerness to computers. After I was grown up and went to the high school
(Rouzbeh High School) our high school heavily insisted on learning new
computer based materials. As a result I earned my ICDL with an average above 90
from International ICDL orginization. Additionally, in the high school, I
managed to be accepted in the first phase of both computer and mathemathics
olympiad. Moreover, me as a member of our team became successful in desigining
a ticketing system for the informatics of the Rouzbeh Institute using Microsoft
Access.

In 2014, I ranked *** out of *** in Iran National University Entrance Exam. I
was admitted to attend computer software engineering in the Amirkabir University
of Technology, the mother of iran industrial universities and oldest industrial
university in Iran. I'm going to be graduated from here in the following year.
In here I started a very interesting and different experience. In the
first two months beginning from my entrance an important competition was ahead
of us. 14th Amirkabir ACM Contest. Suprisingly we ranked first among all the
entrance students and were awarded with 3 quarter gold coins!! We were chosen
as a representative of our university to attend the regional contest at
Sharif University of Technology. In the second semester something cool
happened. We participated in an Fun AI contest held by Sharif University of Technology.
It consisted of two parts. The first phase was held online and only top 30 teams
were admitted to the second phase. Again the same team made it to the Sharif
University of Technology. But this time we were the only accepted team from
our university! But as the time passed by I wonder what do I want to do next?
I found the idea of being connected is pretty awesome. The fact that
a packet can traverse around the world only in 150 ms really shocked me. This
led to my membership in the IoT laborartory of Amirkabir University of Technology.
At first we were trying to turn a lamp on and off using internet but as the time
flied our concerns became bigger and bigger.
\makeletterclosing

\end{document}

%% end of file `template.tex'.
